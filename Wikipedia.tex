\documentclass{beamer}
\usepackage[utf8]{inputenc}
\usetheme{Madrid}
\usecolortheme{default}
\usepackage{graphicx} 
\title{Wikipedia}
\author{Daniel Maćkiewicz}
\date{\today}
\usepackage{amsfonts} 


\begin{document}
\frame{\titlepage}
\begin{frame} 

\frametitle{Spis Treści: \\ Historia \\ Zasady \\ Edytowanie } 
\tableofcontents

\begin{figure} \includegraphics[width=0.25\hsize]{Wikipedia.png} \caption{Logo Wikipedii}\label{fig:wiki} \end{figure}

\end{frame}




 \begin{frame}{Historia} \begin{itemize} \item Wikipedia powstała jako projekt uzupełniający dla Nupedii, darmowej encyklopedii internetowej, 
 
 \end{itemize} \end{frame}

 \begin{frame}{Historia} \begin{itemize}
 
 \item której artykuły mogły być pisane przez każdego, ale miały być sprawdzane i recenzowane przez grupę ekspertów,
 
 \end{itemize} \end{frame}
 
\begin{frame}{Historia} 
\begin{itemize}
 
 \item podobnie jak to się dzieje w przypadku edycji encyklopedii tradycyjnych (książkowych).
 

 
 \end{itemize} \end{frame}


\begin{frame}{Zasady} \begin{itemize} \item Zawartość Wikipedii jest w pewnym zakresie regulowana przez prawo (w szczególności prawo autorskie). 
 
 \end{itemize} \end{frame}
 
 \begin{frame}{Zasady} \begin{itemize} \item Twórców treści Wikipedii obowiązują normy wspólne dla wszystkich wersji językowych Wikipedii – „pięć filarów”. Określają one podstawowe cechy: 
 
 \end{itemize} \end{frame}
 
 \begin{frame}{Zasady} \begin{itemize} \item 1. Wikipedia to encyklopedia,
 \item 2. Wikipedia to neutralny punkt widzenia,
 \item 3. Wikipedia to wolny zbiór wiedzy,
 \item 4. Wikipedia to przestrzeganie netykiety,
 \item 5. Wikipedia to brak sztywnych reguł,
 
 \end{itemize} \end{frame}
 
 \begin{frame}{Zasady} \begin{itemize} \item Początkowo podstawowe zasady nowo tworzonych wersji językowych były tłumaczone z Wikipedii anglojęzycznej, następnie podlegały zmianom w zależności od woli społeczności twórców. 
 
 \end{itemize} \end{frame}
 
 
 
 
 \begin{frame}{Edytowanie} \begin{itemize} 
 \item  Strony Wikipedii z założenia mogą być edytowane w każdej chwili. 
 
 \end{itemize} \end{frame}
 
 \begin{frame}{Edytowanie} \begin{itemize} 
 \item  Możliwość edycji może być jednak czasowo ograniczona (na przykład tylko dla zalogowanych użytkowników) ze względu na częste wandalizmy lub wojny edycyjne. 
 
 \end{itemize} \end{frame}
 
 \begin{frame}{Edytowanie} \begin{itemize} 
 \item Wikipedia nie deklaruje, że którykolwiek artykuł jest „kompletny” lub „skończony” 
 
 \end{itemize} \end{frame}


\end{document}
